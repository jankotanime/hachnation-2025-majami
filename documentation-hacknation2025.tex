\documentclass[a4paper,12pt]{article}
\usepackage[utf8]{inputenc}
\usepackage[polish]{babel}
\usepackage{geometry}
\geometry{a4paper, margin=1in}
\title{Hackathon 2025: Ścieżka Prawa}
\author{}
\date{December 2025}
\begin{document}
\maketitle
\section*{Wstęp}
Projekt powstał podczas HackNation 2025 w ramach tematu \emph{Ścieżka Prawa}. Celem projektu było stworzenie narzędzia wspierającego użytkowników w zrozumieniu i śledzeniu postępów legislacyjnych w Polsce.
\section*{Założenia Projektu}
Na projekt składają się następujące elementy:\begin{enumerate}
\item Przyjazny dla użytkownika frontend zaprojektowany w technologii \textbf{Next.js}.
\item API integrujące się z publicznym API strony rządowej (dane sejmowe) w celu pobierania aktualnych informacji legislacyjnych. API zoptymalizowano pod kątem odświeżania danych.
\item Agent AI połączony z endpointem w języku Python (skrypt) generującym streszczenia ustaw oraz zamieniającym ich treść na prostszy dla użytkownika język.
\item Pasek wskazujący aktualny status ustawy (np. etapy procesu legislacyjnego).\end{enumerate}
\section*{Technologie}
W projekcie wykorzystano następujące technologie:\begin{itemize}
\item \textbf{Spring Boot} - do tworzenia backendu,
\item \textbf{Next.js} - do tworzenia interfejsu użytkownika,
\item \textbf{PostgreSQL} - jako bazę danych.\end{itemize}
\section*{Zespół}
Projekt został wykonany przez trzy osoby:\begin{itemize}
\item Imię i nazwisko 1,
\item Imię i nazwisko 2,
\item Imię i nazwisko 3.\end{itemize}
\section*{Podsumowanie}
Projekt \emph{Ścieżka Prawa} pokazuje, jak technologie mogą wspierać obywateli w poruszaniu się po zawiłościach legislacyjnych. Dzięki zastosowanym rozwiązaniom, użytkownik ma szybki i prosty dostęp do aktualnych danych i ich streszczeń, co zwiększa transparentność i dostępność informacji prawnych.
\end{document}